%%%%%%%%%%%%%%%%%%%%%%%%%%%%%%%%%%%%%%%%%
% Simple Sectioned Essay Template
% LaTeX Template
%
% This template has been downloaded from:
% http://www.latextemplates.com
%
% Note:
% The \lipsum[#] commands throughout this template generate dummy text
% to fill the template out. These commands should all be removed when 
% writing essay content.
%
%%%%%%%%%%%%%%%%%%%%%%%%%%%%%%%%%%%%%%%%%

%----------------------------------------------------------------------------------------
%	PACKAGES AND OTHER DOCUMENT CONFIGURATIONS
%----------------------------------------------------------------------------------------

\documentclass[12pt]{article} % Default font size is 12pt, it can be changed here


\usepackage{polski}
\usepackage[utf8]{inputenc}

\usepackage{geometry} % Required to change the page size to A4
\geometry{a4paper} % Set the page size to be A4 as opposed to the default US Letter

\usepackage{graphicx} % Required for including pictures

\usepackage{float} % Allows putting an [H] in \begin{figure} to specify the exact location of the figure
\usepackage{wrapfig} % Allows in-line images such as the example fish picture

\usepackage{lipsum} % Used for inserting dummy 'Lorem ipsum' text into the template

\linespread{1.2} % Line spacing

%\setlength\parindent{0pt} % Uncomment to remove all indentation from paragraphs

\graphicspath{{Pictures/}} % Specifies the directory where pictures are stored

\begin{document}

%----------------------------------------------------------------------------------------
%	TITLE PAGE
%----------------------------------------------------------------------------------------

\begin{titlepage}

\newcommand{\HRule}{\rule{\linewidth}{0.5mm}} % Defines a new command for the horizontal lines, change thickness here

\center % Center everything on the page

\textsc{\LARGE Politechnika Wrocławska}\\[1.5cm] % Name of your university/college
\textsc{\Large Systemy telemedyczne}\\[0.5cm] % Major heading such as course name
\textsc{\large Kod kursu: TLEU00102P, Termin: Czwartek, 15:15-16:55}\\[0.5cm] % Minor heading such as course title

\HRule \\[0.4cm]
{ \huge \bfseries Teleinformatyczne systemy przywoławcze w placówkach medycznych}\\[0.4cm] % Title of your document
\HRule \\[1.5cm]

\begin{minipage}{0.4\textwidth}
\begin{flushleft} \large
\emph{Autorzy:}\\
Łukasz \textsc{Joksch}(200963) \\
Tomasz \textsc{Kowalik}(200943) \\
Piotr \textsc{Tazbir}(201029)
\end{flushleft}
\end{minipage}
~
\begin{minipage}{0.4\textwidth}
\begin{flushright} \large
\emph{Opiekun:} \\
dr. Edward \textsc{Puchała} % Supervisor's Name
\end{flushright}
\end{minipage}\\[4cm]

{\large \today}\\[3cm] % Date, change the \today to a set date if you want to be precise

%\includegraphics{Logo}\\[1cm] % Include a department/university logo - this will require the graphicx package

\vfill % Fill the rest of the page with whitespace

\end{titlepage}

%----------------------------------------------------------------------------------------
%	TABLE OF CONTENTS
%----------------------------------------------------------------------------------------

\tableofcontents % Include a table of contents

\newpage % Begins the essay on a new page instead of on the same page as the table of contents 

%----------------------------------------------------------------------------------------
%	INTRODUCTION
%----------------------------------------------------------------------------------------

\section{Wstęp} % Major section

Celem niniejszego projektu jest opracowanie zaawansowanego systemu przywoławczego dla placówek medycznych, usprawniającego monitorowanie stanu zdrowia pacjenta oraz umożliwienie szybkiej reakcji personelu w przypadkach nadania sygnału przywoławczego przez chorego. Dzięki gromadzeniu informacji parametrach medycznych pacjenta możliwym będzie odwołanie się do tych danych w przyszłości. Wszystko to dąży do poprawy standardów życia oraz jakości opieki nad pacjentami. Jest dedykowany dla każdej jednostki medycznej, w której zachodzi konieczność całodobowego monitorowania stanu zdrowia pacjenta. \\
Główne założenia tego projektu mogą wydać się rewolucyjne, a wręcz futurystyczne. Należy jednak mieć na uwadze, że to co dziś wydaje się niewykonalnym, za kilka lat może okazać powszechnym standardem.
%------------------------------------------------

\subsection{Systemy przywoławcze, a regulacje prawne} % Sub-section

Poruszany temat jest szczególnie istotny, gdyż zgodnie z rozporządzeniem Ministra Zdrowia z dnia 3. Listopada 2011 w sprawie szpitalnego oddziału ratunkowego obliguje się jednostki medyczne związane z segregacją medyczną do implementacji takich systemów. Również w Rozporządzeniu Ministra Polityki Społecznej z 19. Października 2005 w sprawie domów pomocy społecznej (Dz.U. z 2005 r., nr217, poz. 1837) wymaga się stosowania systemów przyzywowych w tego typu placówkach. Szczególnie istotnym aktem prawnym dotyczącym poruszanych zagadnień jest obowiązująca w Polsce norma PN-EN ISO 11073, która zawiera ogólne wytyczne dotyczące sposobu komunikacji między urządzeniami związanymi ze zdrowiem pacjenta. Wspomniana norma, szczególnie zwraca uwagę aby stosowane rozwiązania pozwalały na gromadzenie informacji o akcji takich urządzeń w postaci historii, z uwzględnieniem możliwości identyfikacji każdego urządzenia. Ponadto, każde zdarzenie powinno być dodatkowo precyzyjnie zarchiwizowane, tj. powinno zawierać datę i godzinę wystąpienia akcji. Niezbędnym jest także warunek zachowania interoperacyjności i zgodności systemów, tak aby stosowane rozwiązanie umożliwiała swobodną wymianę danych między różnymi systemami, w tym także z istniejącą już infrastrukturą. Warunek ten może być spełniony przez wykorzystanie międzynarodowych standardów komunikacji przywoławczych, np. POCSAG. Optymalnym rozwiązaniem wydaje się być funkcjonalność polegająca na możliwości implementacji różnych standardów, tudzież protokołów w zależności od wymagań jednostki medycznej.
W tym miejscu należy wspomnieć o tym, iż o ile polskie prawo zobowiązuje placówki medyczne do stosowania systemów przywoławczych, co więcej wskazuje metody komunikacji w obrębie takiego systemu, jednak w żaden sposób nie są prezentowane wytyczne co do samego systemu. Fakt ten, sprawia, że projektanci oraz konstruktorzy sprzętu oraz oprogramowania mają sporą dowolność, która może wiązać się z niebezpieczeństwem polegającym na wprowadzaniu na rynek produktów słabej jakości. 


%------------------------------------------------

\subsection{Wady i zalety aktualnych rozwiązań} % Sub-section
Aktualnie na rynku istnieje wiele rozwiązań wspierających komunikację między pacjentami a personele szpitala. Jednak ze względu ich komercyjny charakter ich główną cechą wspólną jest indywidualność i nowatorstwo. Niezaprzeczalnie, są to ich ogromne zalety, jednak w ty samym momencie stają się także ich poważnymi wadami. Po pierwsze, częstokroć ile razy konieczna będzie rozbudowa systemu, placówka, jest uzależniona od firmy instalującej system. Hermetyczność biznesowa powoduje, że konkurencyjni dostawcy usług niechętnie dzielą się swoimi rozwiązaniami, wiedza. Takie zjawisko sprzyja monopolizacji rynku, co nie jest korzystnym dla rozwoju gospodarki i tworzeniu nowych technologii. Poza czynnikami stricte biznesowymi i politycznymi, ważnym obszarem jest podejście technologiczne oraz rozwiązania techniczne. Do głównych wad, większości aktualnych systemów możemy zaliczyć:
\begin{enumerate}
\item Stosowanie wielu różnych systemów i języków programowania;
\item Zamknięte oprogramowanie;
\item Ogromne wymagania zasobów sprzętowych;
\item Duże zapotrzebowanie za energię elektryczną;
\item Uzależnienie działania od funkcjonowania; infrastruktury energii elektrycznej(dostarczanej z sieci);
\item Podatność na zakłócenia - utrata integralności i niezawodności pracy;
\item Duże koszty produkcji i utrzymania;
\item Problematyczne aktualizacje;
\item Zamknięte grono osób posiadających wiedzę o systemie.
\end{enumerate}

W związku z występowaniem wielu wad takowych systemów na, bądź co bądź, dość ogólnym poziomie analizy jednym z celi tego projektu będzie próba przedstawienia rozwiązań eliminujących zidentyfikowane problemy.



%----------------------------------------------------------------------------------------
%	MAJOR SECTION 1
%----------------------------------------------------------------------------------------

\section{Założenia projektowe} % Major section

W tym rozdziale przedstawione zostaną główne założenia projektowe. Należy jednak zauważyć, iż poruszanie zagadnienie jest problemem wielopoziomowym, ale też interdyscyplinarnym. Dotyka bowiem wielu obszarów, które z pozoru nie są ze sobą ściśle skorelowane. Jednak po przemyślanej analizie i usystematyzowaniu wymagań i potrzeb wynikających z istnienia ich wszystkich finalnie może powstać sprawnie działający system.

\subsection{Uniwersalność}
Chcąc, choćby częściowo, stworzyć system otwarty ale także taki, dzięki któremu przedsiębiorcy nadal będą mogli zarabiać pieniądze należy stworzyć odpowiedni, optymalny model systemu. Proponujemy czterowarstwowy model przywoławczy:
\begin{itemize}
\item Warstwa Informacji - odpowiada za tworzenie nagłówków przesyłanych informacji, określa ich format i typ(rodzaj). Definiuje strukturę i jakość informacji  dostarczanej przez nośniki informacji określone w warstwie transportowej i Sprzętowej;
\item Warstwa Sprzętowa - określa jaki sprzęt zastosowano w całym systemie;
\item Warstwa Transportowa - definiuje protokoły i sposoby transportu i gromadzenia informacji;
\item Warstwa Aplikacji i Usług - określa zbiór zastosowanego oprogramowania, usług i serwerów.
\end{itemize}

Istotną kwestią, rewolucją względem dotychczas stosowanych rozwiązań, jest budowa systemu przywoławczego opierając się na oprogramowaniu open-source. Dzięki takiemu podejściu chcą rozbudować system nie będzie koniecznym dodatkowe gromadzenie informacji implementacji aktualnego rozwiązania, gdyż będzie ono zawierało sprawdzone, ogólnodostępne oprogramowanie, interfejsy czy protokoły.
%------------------------------------------------

\subsection{Modularyzacja} % Sub-section
Chcąc sprawić, by system przywoławczy był sprawnie funkcjonującym należy dążyć do jego decentralizacji - zarówno pod względem sprzętowym , jak i w ujęciu programowym. Dzięki takiemu rozwiązaniu w przypadku awarii jednego z komponentów nie zostanie zakłócona integralność. Co więcej system powinie być w stanie zlokalizować źródło problemu i w miarę możliwości być w stanie obsłużyć tą sytuację umożliwiając kontynuację pracy bez uszkodzonych elementów infrastruktury.
Taką modularyzację można realizować przez instalację poszczególnych usług czy oprogramowania na kolejnych fizycznych serwerach. W przypadku wymiany informacji zastosowane zostaną Punkty dostępowe sieci WLAN, chcąc zwiększyć dostępność sieci - wystarczy zainstalować kolejne takie urządzenie lub zastosować antenę, zwiększającą zasięg. 
\subsection{Miniaturyzacja} % Sub-sub-section
Całkowicie nowym podejściem jakie zostanie zastosowane w projekcie jest porzucenie kosztownych i zużywających wiele energii elektrycznej serwerów i urządzeń im towarzyszących. W zamian użyte zostaną mikrokomputery Raspberry Pi 3. Ich parametry techniczne pozwalają na implementację zaawansowanych serwerów i usług. Ze względu na bardzo małe koszty związane z zakupem tego sprzętu możliwym jest instalacja poszczególnych usług na kolejnych urządzeniach. Dzięki temu uzyskamy izolację między systemami, która może być użyteczną w przypadku awarii. Ich minimalne zapotrzebowanie na prąd pozwoli na długoterminowe funkcjonowanie rozwiązania na zasilaniu akumulatorowym. Poniżej przedstawiono opis mikrokomputera Raspberry Pi 3:
\begin{itemize}
\item Procesor chipset - Broadcom BCM2837 64-bit
\item Rdzeń	Quad-Core - ARM Cortex A53
\item Taktowanie - 1,2 GHz
\item Pamięć RAM - 1 GB LPDDR2 @ 900 MHz
\item Zasilanie	- 5,1 V 
\item Pobór prądu ~300-400 mAh
\end{itemize}

\section{Koncepcja rozwiązania}
W tym rozdziale zostaną przedstawione informacje dotyczące implementacji każdego z modułu projektu. Zgodnie z wcześniej przyjętym czterowarstwowym modelem - dla każdej warstwy podane zostaną informacje dotyczące jej funkcjonowania i oddziaływania na pozostałe warstwy.

\subsection{Ogólny zarys koncepcyjny}
System przywoławczy opiera się o wykorzystanie dedykowanych urządzeń: pilotów dla pacjentów, modułów przywoławczych dla pacjentów z ograniczonymi funkcjami życiowymi, centrali obsługującej przywołania i trójkolorowego oświetlenia LED wskazującego rodzaj alarmu. Poza rozwiązaniem stricte sprzętowym, zaimplementowana zostanie aplikacja mobilna na najbardziej popularne systemy telefonów(smartfonów).
\\
W nawiązaniu do głównego zamysłu dotyczącego obsługi zdarzeń, gromadzenia informacji o pacjentach i modelu reagowania na te zdarzenia, system będzie realizował możliwie najprostsze ale co ważne - najnowocześniejsze technologie. W chwili gdy pacjent przy pomocy dowolnego narzędzia zasygnalizuje chęć przywołania personelu medycznego - zgłoszenie to natychmiastowo trafi do odpowiedzialnej kadry, system zapisze godzinę, dane pacjenta, oraz rodzaj przywołania. W tej samej chwili na zewnątrz sali szpitalnej zaświeci się lampka przyjmująca barwę odpowiadającą danemu zgłoszeniu. Pozwoli to personelowi na szybką identyfikację odpowiedniej sali. Ponadto na panelu centrali pojawi się także takowa informacja  - z informacją o numerem sali oraz szczegółowe informacje o pacjencie - jego personalia, historia poprzednich przywołań oraz skrót przebiegu choroby(powód hospitalizacji).

\subsection{Rodzaje przywołań oraz ich oznaczenia}

W projekcie przewidziano 3 rodzaje przywołań - alarmów:
\begin{enumerate}
\item krytyczne - przywołanie w przypadku problemów zdrowotnych, zaburzenie funkcji życiowych lub problemy ze zdrowiem, samopoczuciem
\item standardowe - przywołanie związane z obsługą pacjenta i jego codzienna egzystencją, tj. dotyczące szczególnie pomocy pielęgniarskiej, np. pomoc w kąpieli, załatwianiu potrzeb fizjologicznych.
\item komfortowe - dotyczące przywołań związanych z komfortem pobytu w placówce medycznej, np. prośba uruchomienia telewizora, zasłonięcie rolet czy poprawienie pościeli.
\end{enumerate}

W zależności od rodzaju przywołania przewidziano dla każdego z nich odpowiednią barwę koloru. W nawiązaniu do nich projektowane będą kolejne elementy systemu.

\begin{itemize}
\item barwa CZERWONA - dla przywołania krytycznego;
\item barwa ŻÓŁTA - dla przywołania standardowego;
\item barwa ZIELONA - dla przywołania komfortowego.
\end{itemize}
\subsection{Rozwiązania sprzętowe}

Począwszy od terminali pacjentów, należy wspomnieć, iż będą to własne urządzenia skonstruowane specjalnie na potrzeby tego projektu. Oparte o mikrokontroler STM32 z rdzeniem CORTEX. Terminal będzie zawierał 4 duże, ergonomiczne przyciski o barwach odpowiadającym rodzajom wezwania, o których będzie mowa w kolejnych rozdziałach. Jeden z przycisków służyć będzie do odwołania przywołania, np. w chwili omyłkowego wduszenia przycisku. Każdy terminal wyposażony będzie także w moduł WiFi, służący do przesyłania danych do serwera. Przewidziano dwa rodzaje zasilania: bateryjny oraz stacjonarny - zasilanie z sieci energetycznej. 
\\
Terminal centrali będzie tabletem (system ANDROID)z zainstalowanym niezbędnym oprogramowaniem pobierającym informacje o zgłoszeniu z serwera. Do sieci będzie się łączył za pomocą modułu Wifi. Przewidziano także możliwość dźwiękowej sygnalizacji alarmu.
\\
Sercem systemu, a dokładniej, jego trzema głównymi trzonami będą 3 serwery oparte o mikrokomputery Raspbery Pi 3. Każdy z nich będzie posiadał dysk SSD gromadzący dane niezbędne do pracy.
\begin{itemize}
\item serwer obsługujący zdarzenia, powiadamianie odpowiednich central
\item serwer bazodanowy - przechowujący wszystkie zdarzenia w obrębie funkcjonowania systemu
\item serwer backup - gromadzący kopię zapasową całego systemu wykorzystujący technologię RAID5.
\end{itemize}

Dla zapewnienia łączności pomiędzy poszczególnymi elementami rozwiązania zbudowana zostanie zamknięta sieć LAN - INTRANET, uniemożliwiając w ten sposób dostęp do zasobów osób trzecich. Jednak temu zagadnieniu zostanie poświęcony osobny paragraf.

\section{Przepływ informacji, model sieci}

Zgodnie z wcześniejszymi założeniami w projekcie możliwym jest wyszczególnienie warstwy odnoszącej się do przepływu danych. Dla prawidłowego działania całego systemu koniecznym jest istnienie sieci komputerowej. Chcąc zapewnić najwyższy poziom bezpieczeństwa zalecaną jest implementacja sieci przeznaczone wyłącznie dla proponowanego rozwiązania, zarówno odnosząc się do urządzeń sieciowych takich jak przełączniki, rozdzielacze czy punkty dostępowe jak i unikalna, przemyślana  i spójna adresacja IP. Jest to rozwiązanie optymalne, jednak nie jest koniecznym do rozpoczęcia realizacji projektu. W przypadku gdy placówka medyczna posiada już infrastrukturę sieciową możliwym jest zastosowanie rozwiązań typu VLAN czy VPN. W dużej mierze zastosowanie danego podejścia jest ściśle zależne od polityki bezpieczeństwa i zasobów teleinformatycznych danej placówki. W kontekście projektu wymogiem koniecznym do jego realizacji jest zapewnienie niezakłóconego, ciągłego dostępu do sieci informatycznej,  umożliwiającej przesyłanie danych z wykorzystaniem protokołu TCP i UDP dla wszystkich urządzeń.

\subsection{Zastosowane protokoły}

Jak już wcześniej wspomniano, na poziomie warstwy łącza danych modelu OSI/ISO zastosowane będą protokoły TCP i UDP do przesyłania pakietów danych. Jenak na poziomie warstwy aplikacji wykorzystano protokółu HTTP w wersji drugiej do przesyłania informacji stricte związanych z obsługą zdarzeń - np.  identyfikator urządzenia, numer pokoju, rodzaj zgłoszenia. Zdecydowano się właśnie na takie podejście, ze względu na dojrzałość protokołu, bardzo dobre wsparcie techniczne oraz łatwą możliwość śledzenia poprawności generowanych zapytań. Ponadto protokół ten nawiązując połączenie z węzłem docelowym wysyła informację do urządzenia źródłowego, w której infrmuje czy połączenie zostało nawiązane prawidłowo. Dodatkowo nowa (druga) wersja protokołu pozwala na możliwość szyfrowania połączenia na poziomie tego protokołu z pominięciem doatkowych serwerów, tudzież usług szyfrujących.

\section{Zasilanie}
\subsection{Zasilanie awaryjne}

%----------------------------------------------------------------------------------------
%	BIBLIOGRAPHY
%----------------------------------------------------------------------------------------

\begin{thebibliography}{99} % Bibliography - this is intentionally simple in this template

\bibitem[Figueredo and Wolf, 2009]{Figueredo:2009dg}
Figueredo, A.~J. and Wolf, P. S.~A. (2009).
\newblock Assortative pairing and life history strategy - a cross-cultural
  study.
\newblock {\em Human Nature}, 20:317--330.
 
\end{thebibliography}

%----------------------------------------------------------------------------------------

\end{document}